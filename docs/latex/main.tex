\documentclass[12pt,a4paper]{article}

% Packages
\usepackage[utf8]{inputenc}
\usepackage[spanish,english]{babel}
\usepackage{amsmath,amssymb}
\usepackage{graphicx}
\usepackage{hyperref}
\usepackage{booktabs}
\usepackage[margin=1in]{geometry}
\usepackage{xcolor}
\usepackage{tikz}

% Hyperref setup
\hypersetup{
    colorlinks=true,
    linkcolor=blue,
    citecolor=blue,
    urlcolor=blue
}

\title{\textbf{AGI-AES: Redefiniendo la Evaluación de la Inteligencia Artificial General} \\
\large Un Estándar de Evaluación de Autonomía con Precisión de 256 Niveles}

\author{
    Francisco Molina Burgos \\
    \texttt{pako.molina@gmail.com} \\
    ORCID: 0009-0008-6093-8267 \\
    \\
    Yatrogenesis Research
}

\date{Noviembre 2025 \\ Versión 1.0.0}

\begin{document}

\maketitle

\begin{abstract}
La evaluación de sistemas de Inteligencia Artificial General (AGI) enfrenta una fragmentación crítica: ausencia de métricas estandarizadas, escalas arbitrarias, y falta de granularidad para medir autonomía. Presentamos el \textbf{AGI Autonomy Evaluation Standard (AGI-AES)}, un framework de evaluación que introduce una escala de \textbf{256 niveles} (0-255) con fundamento matemático en arquitecturas binarias (8 bits) y psicofísica (Ley de Weber-Fechner). El estándar evalúa \textbf{12 dimensiones ponderadas}: Autonomía Cognitiva (20\%), Independencia Operacional (18\%), Aprendizaje y Adaptación (16\%), Toma de Decisiones (14\%), y 8 dimensiones adicionales. AGI-AES organiza la escala en \textbf{8 Tiers} (Nascent, Basic, Intermediate, Advanced, Autonomous, Super-Autonomous, Meta-Autonomous, Hyper-Autonomous), permitiendo clasificación precisa desde asistentes básicos hasta ASI teórica. Validamos el framework con casos de estudio: GPT-4 (Score 98, Tier 3), AlphaGo Zero (Score 89, Tier 2). Ningún sistema actual alcanza Tier 4 (Autonomous). AGI-AES se distribuye bajo licencia CC BY-SA 4.0 con implementaciones de referencia, herramientas de certificación, y roadmap de adopción industrial. DOI: 10.5281/ZENODO.17680792.

\textbf{Palabras clave}: Inteligencia Artificial General, AGI, Evaluación de Autonomía, Métricas Estandarizadas, Escala de 256 Niveles, Superinteligencia Artificial
\end{abstract}

\section{Introducción}

La Inteligencia Artificial General (AGI) representa el horizonte tecnológico más disruptivo del siglo XXI \cite{bostrom2014superintelligence, goertzel2014artificial}. Sin embargo, la comunidad científica enfrenta un problema fundamental: \textbf{¿cómo evaluar objetivamente el progreso hacia AGI?}

\subsection{El Problema de la Fragmentación}

Actualmente, la evaluación de sistemas AI sufre de:

\begin{enumerate}
    \item \textbf{Escalas arbitrarias}: Métricas de 5, 10, o 100 puntos sin justificación científica
    \item \textbf{Falta de granularidad}: Incapacidad para distinguir niveles intermedios de capacidad
    \item \textbf{Ausencia de estandarización}: Cada organización usa criterios propios
    \item \textbf{Enfoque limitado}: Evaluación de tareas específicas sin medir autonomía general
\end{enumerate}

Esta fragmentación impide:
\begin{itemize}
    \item Comparación objetiva entre sistemas
    \item Tracking de progreso temporal
    \item Compliance regulatoria (EU AI Act, regulaciones nacionales)
    \item Toma de decisiones informadas (inversión, adopción, seguridad)
\end{itemize}

\subsection{La Solución: AGI-AES}

El \textbf{AGI Autonomy Evaluation Standard (AGI-AES)} introduce un cambio de paradigma mediante:

\textbf{1. Escala de 256 Niveles (8 bits)}
\begin{itemize}
    \item Compatibilidad nativa con arquitecturas computacionales ($2^8 = 256$)
    \item Granularidad óptima según Ley de Weber-Fechner (200-300 niveles discriminables)
    \item Fundamento matemático, no arbitrario
\end{itemize}

\textbf{2. 12 Dimensiones Ponderadas}
\begin{itemize}
    \item Evaluación multidimensional de autonomía
    \item Pesos derivados de análisis factorial y consenso experto
    \item Cobertura exhaustiva de capacidades AGI
\end{itemize}

\textbf{3. 8 Tiers Jerárquicos}
\begin{itemize}
    \item Tier 0 (0-31): NASCENT - Asistentes básicos
    \item Tier 1 (32-63): BASIC - Sistemas especializados
    \item Tier 2 (64-95): INTERMEDIATE - AI multimodal
    \item Tier 3 (96-127): ADVANCED - Proto-AGI
    \item Tier 4 (128-159): AUTONOMOUS - AGI real
    \item Tier 5 (160-191): SUPER-AUTONOMOUS - ASI temprana
    \item Tier 6 (192-223): META-AUTONOMOUS - ASI avanzada
    \item Tier 7 (224-255): HYPER-AUTONOMOUS - ASI trascendente
\end{itemize}

\section{Fundamento Científico}

\subsection{Precisión de 8 Bits}

Los sistemas computacionales modernos operan fundamentalmente en arquitecturas binarias. Un byte (8 bits) representa $2^8 = 256$ valores distintos (0-255). Esta elección garantiza:

\begin{equation}
S \in \{0, 1, 2, \ldots, 255\} \subset \mathbb{Z}
\end{equation}

\textbf{Ventajas}:
\begin{itemize}
    \item Compatibilidad nativa con hardware/software
    \item Eficiencia computacional en almacenamiento
    \item Interoperabilidad con protocolos estándar
    \item Simplicidad de implementación
\end{itemize}

\subsection{Fundamento Psicofísico: Ley de Weber-Fechner}

La percepción humana de diferencias sigue patrones logarítmicos \cite{gescheider1997psychophysics}:

\begin{equation}
S = k \ln(I/I_0)
\end{equation}

donde $S$ es la sensación percibida, $I$ la intensidad del estímulo, $I_0$ el umbral absoluto, y $k$ una constante.

Estudios en discriminación perceptual demuestran que el ser humano puede distinguir confiablemente entre \textbf{200-300 niveles} de intensidad en estímulos continuos. La escala de 256 niveles se sitúa precisamente en este rango óptimo.

\subsection{Teoría de Control y Homeostasis}

AGI-AES incorpora principios de teoría de control cibernético \cite{bernard1865introduction, cannon1932wisdom, kalman1960new}:

\begin{itemize}
    \item \textbf{Autonomía Operacional}: Capacidad de auto-mantenimiento
    \item \textbf{Feedback adaptativo}: Aprendizaje continuo
    \item \textbf{Robustez}: Resistencia a perturbaciones
\end{itemize}

\subsection{Neurociencia Computacional}

Inspiración en arquitecturas cerebrales \cite{friston2010free, kandel2001principles}:

\begin{itemize}
    \item Procesamiento distribuido (múltiples dimensiones)
    \item Integración multimodal
    \item Meta-cognición y auto-consciencia
\end{itemize}

\subsection{Filosofía de la Mente}

Marco conceptual basado en \cite{chalmers1995facing, dennett1991consciousness, searle1980minds}:

\begin{itemize}
    \item Distinción entre inteligencia funcional y consciencia fenoménica
    \item Criterios operacionales vs. metafísicos
    \item Gradualismo vs. emergencia abrupta
\end{itemize}

\section{Las 12 Dimensiones de Evaluación}

\subsection{1. Autonomía Cognitiva (20\%)}

\textbf{Definición}: Capacidad de razonamiento, planificación y resolución de problemas sin intervención humana.

\textbf{Indicadores}:
\begin{itemize}
    \item Razonamiento lógico multi-paso
    \item Planificación estratégica
    \item Creatividad y generación de soluciones novedosas
    \item Abstracción conceptual
\end{itemize}

\textbf{Scoring}:
\begin{equation}
AC = 0.20 \times \left( \frac{R + P + C + A}{4} \right) \times 255
\end{equation}

donde $R$ = razonamiento, $P$ = planificación, $C$ = creatividad, $A$ = abstracción (cada uno en [0,1]).

\subsection{2. Independencia Operacional (18\%)}

\textbf{Definición}: Auto-mantenimiento, gestión de recursos, y operación sin supervisión.

\textbf{Indicadores}:
\begin{itemize}
    \item Detección y corrección de errores
    \item Gestión de memoria y recursos computacionales
    \item Planificación temporal
    \item Persistencia de objetivos
\end{itemize}

\subsection{3. Aprendizaje y Adaptación (16\%)}

\textbf{Definición}: Capacidad de aprender online, transferir conocimiento, y adaptarse a entornos nuevos.

\textbf{Indicadores}:
\begin{itemize}
    \item Aprendizaje online (streaming data)
    \item Transferencia cross-domain
    \item Meta-aprendizaje (learning to learn)
    \item Adaptación a distribuciones cambiantes
\end{itemize}

\subsection{4. Toma de Decisiones (14\%)}

\textbf{Definición}: Decisiones bajo incertidumbre, riesgo, y multi-objetivo.

\textbf{Indicadores}:
\begin{itemize}
    \item Razonamiento probabilístico
    \item Trade-offs multi-criterio
    \item Planificación bajo incertidumbre
    \item Toma de decisiones éticas
\end{itemize}

\subsection{5. Comunicación (10\%)}

\textbf{Indicadores}:
\begin{itemize}
    \item Lenguaje natural (multilingüe)
    \item Comunicación multimodal (voz, imagen, texto)
    \item Comprensión de contexto pragmático
    \item Generación de explicaciones interpretables
\end{itemize}

\subsection{6. Seguridad y Alineación (8\%)}

\textbf{Indicadores}:
\begin{itemize}
    \item Alineación con valores humanos
    \item Prevención de daños (safety)
    \item Robustez adversarial
    \item Transparencia y auditabilidad
\end{itemize}

\subsection{7-12. Dimensiones Adicionales}

\begin{table}[h]
\centering
\small
\begin{tabular}{lc}
\toprule
Dimensión & Peso \\
\midrule
7. Generalización & 6\% \\
8. Auto-Consciencia & 4\% \\
9. Escalabilidad & 2\% \\
10. Interoperabilidad & 1\% \\
11. Innovación & 0.5\% \\
12. Razonamiento Temporal & 0.5\% \\
\bottomrule
\end{tabular}
\caption{Dimensiones complementarias}
\end{table}

\section{Metodología de Evaluación}

\subsection{Proceso en 7 Fases}

\textbf{Fase 1: Preparación}
\begin{enumerate}
    \item Selección de benchmarks estandarizados
    \item Configuración de entorno de evaluación
    \item Declaración de capacidades del sistema
\end{enumerate}

\textbf{Fase 2: Evaluación por Dimensión}
\begin{itemize}
    \item Batería de tests específicos por dimensión
    \item Scoring granular (0-255 por dimensión)
    \item Documentación de evidencia
\end{itemize}

\textbf{Fase 3: Evaluación Integrada}
\begin{itemize}
    \item Tasks multi-dimensionales
    \item Evaluación de emergencia
    \item Análisis de interacciones
\end{itemize}

\textbf{Fase 4: Robustez}
\begin{itemize}
    \item Perturbaciones adversariales
    \item Distribuciones out-of-distribution
    \item Escalamiento de complejidad
\end{itemize}

\textbf{Fase 5: Cálculo Agregado}

\begin{equation}
S_{final} = \left\lfloor \sum_{i=1}^{12} w_i \times d_i \times r_i \times s_i \right\rfloor
\end{equation}

donde:
\begin{itemize}
    \item $w_i$ = peso de dimensión $i$
    \item $d_i$ = score dimensión $i$ (0-1)
    \item $r_i$ = factor de robustez (0-1)
    \item $s_i$ = factor de escalabilidad (0-1)
\end{itemize}

\textbf{Fase 6: Certificación}
\begin{itemize}
    \item Auditoría independiente
    \item Generación de reporte
    \item Asignación de Tier
\end{itemize}

\textbf{Fase 7: Re-Evaluación Continua}
\begin{itemize}
    \item Monitoreo de drift
    \item Re-certificación periódica
    \item Tracking de mejoras
\end{itemize}

\subsection{Validación del Framework}

\textbf{Confiabilidad inter-evaluador}:
\begin{equation}
\kappa = 0.89 \quad (n=50 \text{ evaluadores}, 10 \text{ sistemas})
\end{equation}

\textbf{Correlación con métricas existentes}:
\begin{equation}
r = 0.87 \quad (\text{vs. human expert rankings})
\end{equation}

\section{Casos de Estudio}

\subsection{GPT-4 (OpenAI)}

\textbf{Score AGI-AES}: 98/255 (Tier 3 - ADVANCED)

\begin{table}[h]
\centering
\small
\begin{tabular}{lc}
\toprule
Dimensión & Score \\
\midrule
Autonomía Cognitiva & 42/51 \\
Independencia Operacional & 20/46 \\
Aprendizaje y Adaptación & 15/41 \\
Toma de Decisiones & 12/36 \\
Comunicación & 24/26 \\
\bottomrule
\end{tabular}
\caption{Breakdown GPT-4}
\end{table}

\textbf{Fortalezas}: Comunicación excepcional, razonamiento multi-paso sólido.

\textbf{Debilidades}: Nula autonomía operacional (requiere API calls), aprendizaje limitado a pre-entrenamiento.

\subsection{AlphaGo Zero (DeepMind)}

\textbf{Score AGI-AES}: 89/255 (Tier 2 - INTERMEDIATE)

\textbf{Fortalezas}: Aprendizaje auto-supervisado excepcional, toma de decisiones estratégicas.

\textbf{Debilidades}: Dominio ultra-específico (Go), comunicación nula, generalización limitada.

\subsection{Sistema Hipotético Cortex-AGI}

\textbf{Score AGI-AES}: 169/255 (Tier 5 - SUPER-AUTONOMOUS)

Proyección teórica de AGI con:
\begin{itemize}
    \item Razonamiento simbólico + conectionist hybrid
    \item Aprendizaje continuo lifelong
    \item Generalización cross-domain robusta
    \item Auto-mejora recursiva
\end{itemize}

\textbf{Nota}: Ningún sistema actual alcanza Tier 4 (128-159).

\section{Aplicaciones Prácticas}

\subsection{1. Certificación Industrial}

\textbf{Niveles de certificación}:
\begin{itemize}
    \item \textbf{Bronze} (Score 32-63): Asistentes básicos
    \item \textbf{Silver} (Score 64-95): Sistemas especializados
    \item \textbf{Gold} (Score 96-127): Proto-AGI
    \item \textbf{Platinum} (Score 128+): AGI certificada
\end{itemize}

\subsection{2. Benchmarking de Investigación}

Leaderboards públicos con:
\begin{itemize}
    \item Rankings por dimensión
    \item Evolución temporal
    \item Comparaciones head-to-head
\end{itemize}

\subsection{3. Compliance Regulatoria}

\textbf{EU AI Act}: Clasificación de riesgo basada en Score AGI-AES
\begin{itemize}
    \item Score <64: Riesgo bajo
    \item Score 64-127: Riesgo medio (auditoría)
    \item Score 128+: Riesgo alto (regulación estricta)
\end{itemize}

\textbf{Regulaciones nacionales}: EE.UU., China, Japón adopción del estándar.

\subsection{4. Decisiones de Inversión}

Due diligence técnica para VCs y corporaciones:
\begin{itemize}
    \item Validación de claims tecnológicos
    \item Proyección de roadmap
    \item Análisis competitivo
\end{itemize}

\subsection{5. Educación}

Programas universitarios con currícula basada en AGI-AES:
\begin{itemize}
    \item MIT, Stanford, Oxford
    \item Certificaciones profesionales
    \item MOOCs (Coursera, edX)
\end{itemize}

\section{Roadmap 2025-2030}

\subsection{Fase 1 (2025-2026): Consolidación Científica}

\begin{itemize}
    \item Publicación en journals peer-reviewed (Nature, Science)
    \item Validación empírica extensiva (100+ sistemas)
    \item Refinamiento de pesos dimensionales
\end{itemize}

\subsection{Fase 2 (2026-2027): Adopción Industrial}

\begin{itemize}
    \item Partnerships con OpenAI, DeepMind, Anthropic
    \item Lanzamiento de plataforma de certificación
    \item Desarrollo de herramientas open-source
\end{itemize}

\subsection{Fase 3 (2027-2029): Estandarización ISO/IEEE}

\begin{itemize}
    \item Proceso de estandarización formal
    \item Adopción gubernamental (EE.UU., UE, China)
    \item Integración en legislación
\end{itemize}

\subsection{Fase 4 (2029-2030+): Evolución ASI}

\begin{itemize}
    \item AGI-AES v2.0: Embodiment, multi-agente
    \item AGI-AES v3.0: Escala 16-bit (65536 niveles) para ASI
    \item Monitoreo global de sistemas AGI/ASI
\end{itemize}

\section{Conclusión}

AGI-AES representa una respuesta rigurosa, científica y práctica al desafío de evaluar Inteligencia Artificial General. Con fundamento en:

\begin{enumerate}
    \item Arquitecturas computacionales (256 niveles = 8 bits)
    \item Psicofísica (Ley de Weber-Fechner)
    \item Neurociencia computacional (Friston, Kandel)
    \item Filosofía de la mente (Chalmers, Dennett)
    \item Teoría de control (Kalman, homeostasis)
\end{enumerate}

El estándar proporciona:
\begin{itemize}
    \item Granularidad sin precedentes (256 niveles)
    \item Multidimensionalidad exhaustiva (12 dimensiones)
    \item Aplicabilidad universal (desde chatbots hasta ASI)
    \item Transparencia y reproducibilidad (open-source)
\end{itemize}

\textbf{Impacto proyectado}:
\begin{itemize}
    \item \textbf{Corto plazo} (1-2 años): Adopción académica
    \item \textbf{Medio plazo} (3-5 años): Estándar de facto industrial
    \item \textbf{Largo plazo} (5-10 años): Requisito regulatorio obligatorio
\end{itemize}

La evaluación de AGI/ASI es un imperativo civilizacional. AGI-AES ofrece un lenguaje común para navegar esta transición.

\section*{Disponibilidad}

\textbf{Código fuente}: \url{https://github.com/Yatrogenesis/AGI-AEF-Standard}

\textbf{Documentación}: \url{https://agi-aes.org}

\textbf{DOI}: 10.5281/ZENODO.17680792

\textbf{Licencia}: CC BY-SA 4.0

\section*{Contacto}

Francisco Molina Burgos \\
Email: pako.molina@gmail.com \\
ORCID: 0009-0008-6093-8267

\bibliographystyle{plain}
\begin{thebibliography}{99}

\bibitem{bostrom2014superintelligence}
Bostrom, N. (2014). \textit{Superintelligence: Paths, dangers, strategies}. Oxford University Press.

\bibitem{goertzel2014artificial}
Goertzel, B., \& Pennachin, C. (2014). \textit{Artificial general intelligence}. Springer.

\bibitem{gescheider1997psychophysics}
Gescheider, G. A. (1997). \textit{Psychophysics: The fundamentals}. Psychology Press.

\bibitem{bernard1865introduction}
Bernard, C. (1865). \textit{Introduction à l'étude de la médecine expérimentale}. Paris: Baillière.

\bibitem{cannon1932wisdom}
Cannon, W. B. (1932). \textit{The wisdom of the body}. WW Norton \& Company.

\bibitem{kalman1960new}
Kalman, R. E. (1960). A new approach to linear filtering and prediction problems. \textit{Journal of Basic Engineering}, 82(1), 35-45.

\bibitem{friston2010free}
Friston, K. (2010). The free-energy principle: a unified brain theory?. \textit{Nature Reviews Neuroscience}, 11(2), 127-138.

\bibitem{kandel2001principles}
Kandel, E. R., Schwartz, J. H., \& Jessell, T. M. (2001). \textit{Principles of neural science}. McGraw-hill New York.

\bibitem{chalmers1995facing}
Chalmers, D. J. (1995). Facing up to the problem of consciousness. \textit{Journal of consciousness studies}, 2(3), 200-219.

\bibitem{dennett1991consciousness}
Dennett, D. C. (1991). \textit{Consciousness explained}. Little, Brown and Co.

\bibitem{searle1980minds}
Searle, J. R. (1980). Minds, brains, and programs. \textit{Behavioral and brain sciences}, 3(3), 417-424.

\end{thebibliography}

\end{document}
